%% 
%% Copyright 2019-2020 Elsevier Ltd
%% 
%% This file is part of the 'CAS Bundle'.
%% --------------------------------------
%% 
%% It may be distributed under the conditions of the LaTeX Project Public
%% License, either version 1.2 of this license or (at your option) any
%% later version.  The latest version of this license is in
%%    http://www.latex-project.org/lppl.txt
%% and version 1.2 or later is part of all distributions of LaTeX
%% version 1999/12/01 or later.
%% 
%% The list of all files belonging to the 'CAS Bundle' is
%% given in the file `manifest.txt'.
%% 
%% Template article for cas-dc documentclass for 
%% double column output.

%\documentclass[a4paper,fleqn,longmktitle]{cas-dc}
\documentclass[a4paper,fleqn]{cas-dc}

%\usepackage[authoryear,longnamesfirst]{natbib}
%\usepackage[authoryear]{natbib}
\usepackage[numbers]{natbib}

\begin{comment}
\def\tsc#1{\csdef{#1}{\textsc{\lowercase{#1}}\xspace}}
\tsc{WGM}
\tsc{QE}
\tsc{EP}
\tsc{PMS}
\tsc{BEC}
\tsc{DE}
%%%
\end{comment}
%%%Author definitions


\begin{document}
	\let\WriteBookmarks\relax
	\def\floatpagepagefraction{1}
	\def\textpagefraction{.001}
	\shorttitle{Survey of Multimedia Data in Manufacturing}
%	\shortauthors{Yunchao Wang et~al.}
	
	\title [mode = title]{Visualization and Visual Analysis of Multimedia Data in Manufacturing: A Survey}                      
	%\tnotemark[1,2]
	
	\begin{comment}
	\tnotetext[1]{This document is the results of the research
	project funded by the National Science Foundation.}
	
	\tnotetext[2]{The second title footnote which is a longer text matter
	to fill through the whole text width and overflow into
	another line in the footnotes area of the first page.}
	
	\end{comment}
	
	
%	\author{Yunchao Wang}
%	\author {Zihao Zhu}
%	\author{Lei Wang}
%	\author{Guodao Sun*}
%	\ead{godoor.sun@zjut.edu.cn}
%	\author{Ronghua Liang}
%	\address{College of Computer Science, Zhejiang University of Technology, Hangzhou, 310023}
	

%	\author{Guodao Sun{mycorrespondingauthor}}
%\cortext[mycorrespondingauthor]{Corresponding author}
	
	\begin{comment}
	\author[1,3]{CV Radhakrishnan}[type=editor,
	auid=000,bioid=1,
	prefix=Sir,
	role=Researcher,
	orcid=0000-0001-7511-2910]
	\cormark[1]
	\fnmark[1]
	\ead{cvr_1@tug.org.in}
	\ead[url]{www.cvr.cc, cvr@sayahna.org}
	
	\credit{Conceptualization of this study, Methodology, Software}
	
	\address[1]{Elsevier B.V., Radarweg 29, 1043 NX Amsterdam, The Netherlands}
	
	\cortext[cor1]{Corresponding author}
	
	\cortext[cor2]{Principal corresponding author}
	\fntext[fn1]{This is the first author footnote. but is common to third
	author as well.}
	\fntext[fn2]{Another author footnote, this is a very long footnote and
	it should be a really long footnote. But this footnote is not yet
	sufficiently long enough to make two lines of footnote text.}
	
	\nonumnote{This note has no numbers. In this work we demonstrate $a_b$
	the formation Y\_1 of a new type of polariton on the interface
	between a cuprous oxide slab and a polystyrene micro-sphere placed
	on the slab.
	}
	\end{comment}
	
	
	\begin{keywords}
		Visualization \sep Visual Analysis \sep Manufacturing \sep Multi-media data \sep Industry 4.0
	\end{keywords}
	
	
	\begin{abstract}
		%随着生产技术和社会需求的发展,制造业的门类不断完善。而传感器和计算机的使用,使得制造业中多媒体数据的收集越来越方便。根据多媒体数据的类型进行针对性的快速细致的分析可以在制造业全流程的不同阶段做出及时的决策。可视化与可视分析因其直观性和可交互性,在数据的理解、展示和分析上展现了强大的能力,被频繁应用在制造业多媒体数据分析中。在本文中,我们提出了一个可视化与可视分析的文献综述,专门为制造业多媒体数据。我们根据可视化技术、交互分析方法和应用领域对现有研究进行分类。在制造业研究项目的具体例子中,我们讨论了可视化和可视化分析应用于不同类型的多媒体数据时的差异。最后,我们对现有挑战做了总结,指出未来研究方向。
		With the development of production technology and social needs, sectors of manufacturing are constantly improving. The use of sensors and computers has made it increasingly convenient to collect multimedia data in manufacturing. Targeted, rapid and detailed analysis based on the type of multimedia data can make timely decisions at different stages of the entire manufacturing process. Visualization and visual analytics are frequently adopted in manufacturing multimedia data analysis because of their powerful ability to understand, present and analyze data in an intuitive and interactive way. In this paper, we present a literature review of visualization and visual analytics specifically for manufacturing multimedia data. We classify existing researches according to visualization techniques, interaction analysis methods and application areas. We discuss the differences when visualization and visual analytics are applied to different types of multimedia data in the context of specific examples from manufacturing research projects. Finally, we summarize the existing challenges and prospect the future research directions.
	\end{abstract}
	
	\maketitle
	
%工业4.0的核心是智能生产技术和智能生产模式,旨在通过“物联网”和“务联网”,把产品、机器、资源和人联系在一起,推动各环节数据共享,实现产品全生命周期和全制造流程的数字化。
%The core of Industry 4.0 is smart production technology and model. It aims to connect products, machines, resources and people through the \textit{Internet of Things} (IoT)~\cite{gubbi2013internet} and \textit{Internet of Services} (IoS)~\cite{cardoso2008service}, promote data sharing in all aspects, and realize the digitalization of the whole product life cycle and the whole manufacturing process.

%通过广泛的文献回顾,添加了最近出现的新方法,并按照新的角度对主题进行重新分类。
 


%  介绍
\section{Introduction}


% 相关综述和搜索文献方法
\section{Related Surveys And Methodology}


% 分类法
\section{Taxonomy}


% 可视化技术
\section{Visual Analysis Techniques}


% 应用
\section{Application}

% 

\bibliographystyle{cas-model2-names}
\bibliography{SISI_dc}

\end{document}

